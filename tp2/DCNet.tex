\begin{Interfaz}

  \textbf{usa}: \tadNombre{red, conj($\alpha$), itConj($\alpha$), lista($\alpha$), itLista($\alpha$), dicc$_{UNIV}$($\kappa$, $\sigma$), dicc$_{LOG}$($\kappa$, $\sigma$), cola$_{LOG}$($\alpha$)}.
  
  \textbf{se explica con}: \tadNombre{DCNet}.

  \textbf{g\'eneros}: \TipoVariable{dcnet}.

  \Titulos{Operaciones de DCNet}

  \InterfazFuncion{Red}{\In{d}{dcnet}}{red}
  {alias($res$ =$_{obs}$ red($d$))}%
  [O(1)]
  [devuelve la red asociada.]
  [res no es modificable.]

  \InterfazFuncion{CaminoRecorrido}{\In{d}{dcnet}, \In{p}{IDpaquete}}{lista(hostname)}
  [IDpaqueteEnTransito?($d$, $p$)]
  {$res$ =$_{obs}$ dameSecuDeHostnames(caminoRecorrido($d$, damePaquete($p$)))}%
  [O(n * log(k))]
  [devuelve el camino recorrido desde el origen hasta el actual.]
  [res se devuelve por copia.]
  
  \InterfazFuncion{CantidadEnviados}{\In{d}{dcnet}, \In{c}{hostname}}{nat}
  [$c$ $\in$ dameHostnames(computadoras(red($d$)))]
  {$res$ =$_{obs}$ cantidadEnviados($d$, dameCompu($c$))}%
  [O(L)]
  [devuelve la cantidad de paquetes enviados por la computadora.]
  []

  \InterfazFuncion{EnEspera}{\In{d}{dcnet}, \In{c}{hostname}}{conj(paquete)}
  [$c$ $\in$ dameHostnames(computadoras(red($d$)))]
  {alias($res$ =$_{obs}$ enEspera($d$, dameCompu($c$)))}%
  [O(L)]
  [devuelve los paquetes en la cola de la computadora.]
  [res no es modificable.]
  
  \InterfazFuncion{IniciarDCNet}{\In{r}{red}}{dcnet}
  [true]
  {$res$ =$_{obs}$ iniciarDCNet($r$)}%
  [O(N*n*(L+n)]
  [crea una nueva Dcnet.]
  [la red se agrega por copia.]
    
  \InterfazFuncion{CrearPaquete}{\Inout{d}{dcnet}, \In{p}{paquete}}{}
  [$d$ = $d_0$ $\wedge$ $\neg$( ($\exists$ $p\sp{\prime}$: paquete) (paqueteEnTransito?($d$, $p\sp{\prime}$) $\wedge$ id($p\sp{\prime}$) = id($p$) ) $\wedge$ \newline
  origen($p$) $\in$ computadoras(red($d$)) $\wedge_L$ destino($p$) $\in$ computadoras(red($d$)) $\wedge_L$ hayCamino?(red($d$), origen($p$), destino($p$))]
  {$d$ $=_{obs}$ crearPaquete($d_0$, $p$)}%
  [O(L * log(k))]
  [agrega un paquete a la red.]
  [el paquete se agrega por copia.]

\newpage

  \InterfazFuncion{AvanzarSegundo}{\Inout{d}{dcnet}}{}
  [$d$ = $d_0$]
  {$d$ =$_{obs}$ avanzarSegundo($d_0$)}%
  [O(n * (L + log(k)))]
  [realiza los movimientos de paquetes correspondientes, aplicando los cambios necesarios a la dcnet.]
  []
  
  \InterfazFuncion{PaqueteEnTransito?}{\In{d}{dcnet}, \In{p}{IDpaquete}}{bool}
  [true]
  {$res$ =$_{obs}$ IDpaqueteEnTransito?($d$, $p$)}%
  [O(n * k)]
  [indica si el paquete esta en alguna de las colas dado el ID.]
  []
  
  \InterfazFuncion{LaQueMasEnvio}{\In{d}{dcnet}}{hostname}
  [true]
  {$res$ =$_{obs}$ laQueMasEnvio($d$).ip}%
  [O(1)]
  [devuelve la computadora que m\'as paquetes envi\'o.]
  [res se devuelve por copia.]  
  
  \InterfazFuncion{$\bullet = \bullet$}{\In{d_1}{dcnet}, \In{d_2}{dcnet}}{bool}
  [true]
  {$res$ =$_{obs}$ ($d_1$ =$_{obs}$ $d_2$}%
  [O(n*n * ( L + n*n + m + log(k) ) + n*(m*m + L))]
  [indica si dos dcnet son iguales.]
  []  
  
  \textbf{donde:} \newline  
   \TipoVariable{hostname} es \TipoVariable{string}, \newline
   \TipoVariable{interfaz} es \TipoVariable{nat}, \newline
   \TipoVariable{IDpaquete} es \TipoVariable{nat}, \newline
   \TipoVariable{compu} es \TipoVariable{tupla}< ip: \TipoVariable{hostname}, interfaces: \TipoVariable{conj}(\TipoVariable{interfaz})>, \newline
   \TipoVariable{paquete} es \TipoVariable{tupla}< id: \TipoVariable{IDpaquete}, prioridad: \TipoVariable{nat}, origen: \TipoVariable{hostname}, destino: \TipoVariable{hostname} >.
   
\end{Interfaz}

\newpage

\textbf{Especificaci\'on de las operaciones auxiliares utilizadas en la interfaz (no exportadas)}

\begin{tad}{\tadNombre{DCNet extendida}}

\tadExtiende{\tadNombre{DCNet}}

\tadOtrasOperaciones 

\tadOperacion{damehostnames}{conj(compu)}{conj(hostname)}{}
\tadOperacion{dameCompu}{dcnet /d, hostname /s}{compu}{$s$ $\in$ dameHostnames(computadoras(red($d$)))}
\tadOperacion{auxDameCompu}{hostname /s, conj(compu) /cc}{compu}{}
\tadOperacion{dameSecuDeHostnames}{secu(compu)}{secu(hostname)}{}
\tadOperacion{IDpaqueteEnTransito?}{dcnet /d, IDpaquete /p}{bool}{}
\tadOperacion{damePaquete}{dcnet /d, IDpaquete /p}{paquete}{IDpaqueteEnTransito?($d$,$p$)}
\tadOperacion{dameIDpaquetes}{conj(paquete)}{conj(IDpaquete)}{}

\tadAxiomas[\paratodo{dcnet}{d}, \paratodo{hostname}{s}, \paratodo{IDpaquete}{p}, \paratodo{conj(compu)}{cc}, , \paratodo{secu(compu)}{secu}, \paratodo{conj(paquete)}{cp}, ]

\tadAxioma{dameHostnames($cc$)}
{ \IF {vacio?($cc$)} 
THEN {$\emptyset$}
ELSE {Ag( ip(dameUno($cc$)), dameHostnames(sinUno($cc$)) )}
FI}

\tadAxioma{dameCompu($d$, $s$)}{auxDameCompu($s$, computadoras(red(($d$)))}

\tadAxioma{auxDameCompu($s$, $cc$)}
{ \IF {ip(dameUno($cc$)) = $s$} 
THEN {dameUno($cc$)}
ELSE {auxDameCompu($s$, sinUno($cc$))}
FI}

\tadAxioma{dameSecuDeHostnames($secu$)}
{ \IF {vacia?($secu$)} 
THEN {$<>$}
ELSE {ip(prim($secu$)) $\bullet$ dameSecuDeHostnames(fin($secu$)) }
FI}

\tadAxioma{IDpaqueteEnTransito?($d$, $p$)}{auxIDpaqueteEnTransito($d$, computadoras(red($d$)), $p$)}

\tadAxioma{auxIDpaqueteEnTransito($d$, $cc$, $p$)}
{ \IF {vacio?($cc$)} 
THEN {false}
ELSE { \IF {$p$ $\in$ dameIDpaquetes(enEspera(dameUno($cc$)))}
		THEN {true}
		ELSE { auxIDpaqueteEnTransito($d$, sinUno($cc$), $p$)}
		FI}
FI}

\tadAxioma{dameIDpaquetes($cp$)}
{ \IF {vacio?($cp$)} 
THEN {$\emptyset$}
ELSE {Ag( id(dameUno($cp$)), dameIDpaquetes(sinUno($cp$)) )}
FI}

\end{tad}

\newpage

\begin{Representacion}

\begin{Estructura}{dcnet}[estr\_dcnet]

\begin{Tupla}[estr\_dcnet]
	\tupItem{red}{red} \newline \nomoreitems
	\tupItem{computadoras}{dicc(hostname, X)} \newline \nomoreitems
	\tupItem{porHostname}{dicc$_{UNIV}$ (hostname, itDicc(hostname, X))} \newline \nomoreitems
	\tupItem{conMasEnvios}{itDicc(hostname, X)} \newline \nomoreitems
	\tupItem{caminos}{arreglo\_dimensionable de arreglo\_dimensionable de lista(hostname)} \nomoreitems
\end{Tupla}

\begin{Tupla}[X]
	\tupItem{indice}{nat} \newline \nomoreitems
	\tupItem{paquetes}{conj(paquete)} \newline \nomoreitems
	\tupItem{cola}{cola$_{LOG}$(itConj(paquete))} \newline \nomoreitems
	\tupItem{paqPorID}{dicc$_{LOG}$ (IDpaquete, itConj(paquete))} \newline \nomoreitems
	\tupItem{cantEnvios}{nat} \nomoreitems
\end{Tupla} 

\end{Estructura}

\comentario{REP en Castellano: \newline
1: Las compus de Red son las compus de DCNet.\newline
2: PorHostname y computadoras tienen el mismo conjunto de claves. \newline
3: PorHostname permite acceder a los datos de todas las computadoras a trav\'es de iteradores.\newline
4: Los indices de las computadoras van de 0 a n-1. \newline
5: Los indices no se repiten.\newline
6: ConMasEnvios es un interador a la computadora con mayor cant de envios.\newline
7: La matriz de caminos es de n x n.\newline
8: En la matriz caminos[i][j] se guarda uno de los caminos minimos de la red correspondiente al origen y destino correspondientes a los indices i, j, respectivamente. Si no hay, se guarda una lista vacia.\newline
9: Las claves del diccionario paquetesPorID son los ID del conjunto paquetes.\newline
10: El conjunto de paquetes y la cola de prioridad tienen el mismo tamano.\newline 
11: La cola ordena los paquetes por prioridad. (usando los observadores del TAD Cola de Prioridad Alaternativa adjunto). \newline
Para todos los paquetes de una computadora:\newline
12: El origen y el destino estan entre las computadoras de la dcnet.\newline
13: El origen y el destino son distintos.\newline
14: Hay un camino posible entre el origen y el destino.\newline
15: La computadora actual esta en el camino minimo entre el origen y el destino.\newline
16: El id es unico.\newline
17: Son accesibles por el dicc usando su ID.\newline
}

\newpage

\tadAlinearFunciones{REP}{estr\_dcnet /d}

\Rep[estr\_dcnet]{
	\comentario{1} dameHostnames(computadoras($e$.red)) = claves ($e$.computadoras) $\wedge$ 
	\comentario{2} claves ($e$.computadoras) = claves ($e$.porHostname) $\wedge$ 
	\comentario{3} ($\forall$ $c$: hostname, $c$ $\in$ claves($e$.porHostname)) (
	SiguienteClave(obtener($e$.porHostname, $c$)) = $c$ $\wedge$
	SiguienteSignificado(obtener($e$.porHostname, $c$)) = obtener($e$.computadoras, $c$) ) $\wedge$ \newline
	($\forall$ $c$: hostname, $c$ $\in$ claves($e$.computadoras))
	\comentario{4} 0 $<$ obtener($e$.computadoras, $c$).indice $<$ $\#$claves($e$.computadoras)-1 $\wedge$ obtener($e$.computadoras, $c$).indice = ordenLexicografico($c$, claves($e$.computadoras)) $\wedge$
	\comentario{5} $\neg$($\exists$ $c\sp{\prime}$: hostname, $c\sp{\prime}$ $\in$ claves($e$.computadoras), $c$ $\neq$ $c\sp{\prime}$) obtener($e$.computadoras, $c\sp{\prime}$).indice = obtener($e$.computadoras, $c$).indice $\wedge$ 
	\comentario{6} $\neg$($\exists$ $c\sp{\prime}$: hostname, $c\sp{\prime}$ $\in$ claves($e$.computadoras), $c$ $\neq$ $c\sp{\prime}$) obtener($e$.computadoras, $c\sp{\prime}$).cantEnvios > SiguienteSignificado($e$.conMasEnvios).cantEnvios $\wedge$ 
	\comentario{7} tam($e$.caminos) = $\#$claves($e$.computadoras) $\wedge_L$ 
	($\forall$ $i$: nat, 0 $<$ i $<$ $\#$claves($e$.computadoras)-1 ) tam($e$.caminos[$i$]) = $\#$claves($e$.computadoras) $\wedge$ 
	\comentario{8} ($\forall$ $c1, c2$: hostname, $c1, c2$ $\in$ claves($e$.porHostname)) \newline
	$\neg$ $\emptyset$? (caminosMinimos($e$.red, dameCompu($c1$), dameCompu($c2$))) $\Rightarrow$ 
	$e$.caminos[obtener($e$.computadoras, $c1$).indice][obtener($e$.computadoras, $c2$).indice] = dameUno(caminosMinimos($e$.red, dameCompu($c1$), dameCompu($c2$))) $\wedge$ \newline
	$\emptyset$? (caminosMinimos($e$.red, dameCompu($c1$), dameCompu($c2$))) $\Rightarrow$ 
	$e$.caminos[obtener($e$.computadoras, $c1$).indice][obtener($e$.computadoras, $c2$).indice] = Vacia() $\wedge$ \newline
	($\forall$ $c$: hostname, $c$ $\in$ claves($e$.computadoras)) (
	\comentario{9} dameIDpaquetes(obtener($e$.computadoras, $c$).paquetes) = claves(obtener($e$.computadoras, $c$).paquetesPorID) $\wedge$ 
	\comentario{10} $\#$(obtener($e$.computadoras, $c$).paquetes) = $\#$(obtener($e$.computadoras, $c$).cola) $\wedge$ 
	\comentario{11} vacia?(obtener($e$.computadoras, $c$).cola) = $\emptyset$?(obtener($e$.computadoras, $c$).paquetes) $\wedge$ \newline Siguiente($\Pi_2$(proximo(obtener($e$.computadoras, $c$).cola))) $\in$ obtener($e$.computadoras, $c$).paquetes \newline $\wedge$ 
	$\neg$($\exists$ $p\sp{\prime}$: paquete, $p\sp{\prime}$ $\in$ obtener($e$.computadoras, $c$).paquetes) $p\sp{\prime}$.prioridad $<$ Siguiente($\Pi_2$(proximo(obtener($e$.computadoras, $c$).cola))).prioridad $\wedge$ \newline
	$\Pi_1$(proximo(obtener($e$.computadoras, $c$).cola)) = Siguiente($\Pi_2$(proximo(obtener($e$.computadoras, $c$).cola))).prioridad $\wedge$ \newline
	desencolar(obtener($e$.computadoras, $c$).cola) = armarCola(obtener($e$.computadoras, $c$).paquetes - $\{$Siguiente($\Pi_2$(proximo(obtener($e$.computadoras, $c$).cola)))$\}$ $\wedge$
	\newline
	($\forall$ $p$: paquete, $p$ $\in$ obtener($e$.computadoras, $c$).paquetes) (
	\comentario{12} origen($p$).ip $\in$ claves ($e$.computadoras) $\wedge$ 
destino($p$).ip $\in$ claves ($e$.computadoras) $\wedge$ 
	\comentario{13} origen($p$).ip $\neq$ destino($p$).ip $\wedge$ 
	\comentario{14} hayCamino?($e$.red, origen($p$), destino($p$)) $\wedge$ 
	\comentario{15} esta? ($c$, caminos[obtener($e$.computadoras, origen($p$).ip)][obtener($e$.computadoras, destino($p$).ip) ) $\wedge$
	\comentario{16} ($\forall$ $c\sp{\prime}$: hostname, $c\sp{\prime}$ $\in$ claves($e$.computadoras), $c\sp{\prime}$ $\neq$ $c$ )  $\neg$($\exists$ $p\sp{\prime}$: paquete, $p\sp{\prime}$ $\in$ obtener($e$.computadoras, $c\sp{\prime}$).paquetes, $p$ $\neq$ $p\sp{\prime}$) $p$.id = $p\sp{\prime}$.id 
	\comentario{17} 
	definido?(obtener($e$.computadoras, $c$).paquetesPorID, $p$.id) $\wedge_L$ \newline
	Siguiente(obtener(obtener($e$.computadoras, $c$).paquetesPorID,  $p$.id)) = $p$
}

\textbf{}

\textbf{Especificaci\'on de las funciones auxiliares utilizadas en Rep}

\tadAlinearFunciones{armarCola}{conj(paquetes)} 

\tadOperacion{armarCola}{conj(paquete)}{cola(paquete)}{}

\tadAxiomas[\paratodo{conj(paquete)}{cc}]

\tadAxioma{armarCola($cc$)}{
{ \IF {$\emptyset$?($cc$)} 
THEN {Vacia()}
ELSE {encolar(dameUno($cc$).prioridad, dameUno($cc$), armarCola(sinUno($cc$)))}
FI}
}

\textbf{}

\AbsFc[estr\_dcnet]{dcnet}[e]{ $d$ $|$ 
red($d$) = $e$.red $\wedge$ \newline
($\forall$ $c$: compu, $c$ $\in$ computadoras(red($d$))) ( \newline
cantidadEnviados($d$, $c$) = obtener($e$.computadoras, $c$.ip).cantEnvios $\wedge$ \newline
enEspera($d$, $c$) = obtener($e$.computadoras, $c$.ip).paquetes $\wedge$ \newline
($\forall$ $p$: paquete, $p$ $\in$  obtener($e$.computadoras, $c$.ip).paquetes ) 
caminoRecorrido ($d$, $p$) = $e$.caminos[obtener($e$.computadoras, origen($p$).ip).indice][obtener($e$.computadoras, $c$.ip).indice] ) 
}

\end{Representacion}

\begin{Algoritmos}

\begin{algorithm}
\caption{Implementaci\'on de Red}
\begin{algorithmic}[0]
\Function{iRed}{\In{d}{estr\_dcnet}}{$\rightarrow$ res: Red}
	\State res $\gets$ d.red \Comment{O(1)}
\EndFunction \Comment{O(1)}
\end{algorithmic}
\end{algorithm}

\begin{algorithm}
\caption{Implementaci\'on de CaminoRecorrido}
\begin{algorithmic}[0]
\Function{iCaminoRecorrido}{\In{d}{estr\_dcnet}, \In{p}{IDPaquete}}{$\rightarrow$ res: lista(hostname)}
	\State itCompu $\gets$ CrearIt(d.computadoras) \Comment{O(1)}
	\State yaEncontrado $\gets$ FALSE \Comment{O(1)}
	\While{HaySiguiente(itCompu) $\&\&$ $\neg$yaEncontrado} \Comment{Guarda: O(1)} \Comment{Se repite a lo sumo n veces} \Comment{O(n)}
		\If{Definido?(SiguienteSignificado(itCompu).paqPorID, p)} \Comment{O(log(k))} 
			\State paquete $\gets$ Significado(SiguienteSignificado(itCompu).paqPorID, p) \Comment{O(1)}
			\State yaEncontrado $\gets$ TRUE \Comment{O(1)}
		\Else
			\State Avanzar(itCompu) \Comment{O(1)}
		\EndIf
	\EndWhile
	
	\State res $\gets$ caminos[Significado(d.computadoras, $\pi$3(paquete)).indice][SiguienteSignificado(itCompu).indice] \Comment{O(1) + O(n) + O(1)}
\EndFunction \Comment{O( n * log(k) )}
\end{algorithmic}
\end{algorithm}

\begin{algorithm}
\caption{Implementaci\'on de paquetes enviados}
\begin{algorithmic}[0]
\Function{iCantidadEnviados}{\In{d}{estr\_dcnet}, \In{c}{hostname}}{$\rightarrow res$: $nat$}
	\State it $\gets$ Significado(d.porHostname, c) \Comment{O(L)}
	\State res $\gets$ SiguienteSignificado(it).cantEnvios\Comment {O(1)}
\EndFunction \Comment{O(L)}
\end{algorithmic}
\end{algorithm}

\begin{algorithm}
\caption{Implementaci\'on EnEspera}
\begin{algorithmic}[0]
\Function{iEnEspera}{\In{d}{estr\_dcnet}, \In{c}{hostname}}{$\rightarrow res: estr$}
	\State it $\gets$ Significado(d.porHostname, c) \Comment{O(L)}
	\State res $\gets$ SiguienteSignificado(it).paquetes \Comment{O(1)}
\EndFunction \Comment{O(L)}
\end{algorithmic}
\end{algorithm}


\begin{algorithm}
\caption{Implementaci\'on de iniciarDCNet}
\begin{algorithmic}[0]
\Function{iIniciarDCNet}{\In{r}{red}}{$\rightarrow res$: $estr\_dcnet$}
	\comentario{creo un diccionario lineal}
	\State diccCompus $\gets$ Vacio()  \Comment{O(1)}
	\comentario{creo un diccionario universal(trie)}
	\State diccHostname $\gets$ Vacio()  \Comment{O(1)}
\comentario{creo una lista vac\'ia donde voy a guardar los hostnames y ordenarlos}
	\State listaComp $\gets$ Vacia() \Comment{O(1)}
	\State itHostname $\gets$  CrearIt(Computadoras(r)) \Comment{O(1)}
	\State masEnvios $\gets$ Siguiente(itHostname) \Comment{O(1)}
	\comentario{$\#$Computadoras(r)) \'o	O(n)}
	\While{HaySiguiente(itHostname)} \Comment{O(n) + O(1)}
\comentario{agrego el hostname a la lista de computadoras}
		\State AgregarAtras(listaComp, Siguiente(itHostname)) \Comment{O(L)}
		\comentario{Inicia el índice como cero, mas adelante les pondremos valor} 
		\State X $\gets$ <0, Vacio(), Vacio(), Vacio(), 0> \Comment{O(1) + O(1) + O(1) + O(1) + O(1)}
		\comentario{ver complejidad}
		\State itX $\gets$ DefinirRapido(diccCompus, Siguiente(itHostname), X)  \Comment{O(copy(hostname)) + O(copy(X))} 
		\comentario{ver complejidad}
		\State Definir(diccHostname, Siguiente(itHostname), itX)  \Comment{O(L) + O(copy(X))}
		\State Avanzar(itHostname) \Comment{O(1)}
	\EndWhile
	\State itPC $\gets$ CrearIt(diccCompus) \Comment{O(1)}
	\State itPC2 $\gets$ CrearIt(diccCompus) \Comment{O(1)}				 	 	
	\State n $\gets$ \#Claves(diccCompus) \Comment{O(1)}
	\State arrayCaminos $\gets$ CrearArreglo(n) \Comment{O(n)}
	\comentario{voy a crear un arreglo en cada posicion de arrayCaminos, el cual va a tener el minimo camino}
	\comentario{$\#$Computadoras(r)) \'o	O(n)}
	\While{HaySiguiente(itPC)} \Comment{O($\#$Computadoras(r))}
		\State arrayDestinos $\gets$ CrearArreglo(n) \Comment{O(n)}
		\comentario{\comentario{$\#$Computadoras(r)) \'o	O(n)}}
		\While{HaySiguiente(itPC2)} \Comment{O($\#$Computadoras(r))}
			\State ConjCaminos $\gets$ CaminosMinimos(r, SiguienteClave(itPC), SiguienteClave(itPC2)) \Comment{O(n*L)}
			\State itConj $\gets$ CrearIt(ConjCaminos) \Comment{O(1)}
			\comentario{de todos los caminos minimos me quedo con uno}
			\If{HaySiguiente(itConj)}   \Comment{O(1)}
				\State arrayDestinos[SiguienteSignificado(itPC2).indice] $\gets$ Siguiente(itConj) \Comment{O(1)}
			\Else 
				\comentario{si no hay camino, creo una lista vacia}
				\State arrayDestinos[SiguienteSignificado(itPC2).indice] $\gets$ Vacia() \Comment{O(1)}
			\EndIf
			
				\algstore {iniciar}
	
\end{algorithmic}
\end{algorithm}

\begin{algorithm}
\begin{algorithmic}[0]

	\algrestore {iniciar}
	
			\State Avanzar(itPC2) \Comment{O(1)}
		\EndWhile
		\State arrayCaminos[SiguienteSignificado(itPC).indice] $\gets$ arrayDestinos \Comment{O(1)}
		\State Avanzar(itPC) \Comment{O(1)}
	\EndWhile

\comentario{inicio el Indice en 0}
	\State indice $\gets$  0 \Comment{O(1)}	
	\While{indice < \#Claves(Computadoras(r))} \Comment{Guarda: O(n)} \Comment{se ejecuta n veces} \Comment{O(n)}
\comentario{busco el m\'inimo de la lista de hostnames (por \'orden alfab\'etico}
		\State itHostnames $\gets$ CrearIt(listaComp) \Comment{O(1)}
		\State min $\gets$ Copiar(Siguiente(itHostnames)) \Comment{O(L)}
		\State Avanzar(itHostnames) \Comment{O(1)}
		\While{HaySiguiente(itHostnames)}
			\If{min < Siguiente(itHostnames)} \Comment{O(L)}
				\State min $\gets$ Copiar(Siguiente(itHostnames)) \Comment{O(L)}
			\EndIf
			\State Avanzar(itHostnames) \Comment{O(1)}
		\EndWhile
		\State Significado(diccCompus, min).indice = indice \Comment{O(n)}
\comentario{creo un iterador de la lista para eliminar el minimo que ya use}
		\State itHostnames $\gets$ CrearIt(listaComp) \Comment{O(1)}
		\State noElimine $\gets$ TRUE
		\While{HaySiguiente(itHostnames) $\&\&$ noElimine} \Comment{se ejecuta a lo sumo n veces} \Comment{O(n)}
			\If{Siguiente(itHostnames) == min} \Comment{O(L)}
				\State EliminarSiguiente(itHostnames) \Comment{O(1)}
				\State noElimine $\gets$ FALSE \Comment{O(1)}
			\EndIf
			\State Avanzar(itHostnames) \Comment{O(1)}
		\EndWhile
		\State indice $\gets$ indice + 1 \Comment{O(1)}
	\EndWhile
	
	
	\State res $\gets$ < Copiar(r), diccCompus, diccHostname, masEnvios, arrayCaminos > \Comment{O(Copiar(r)) + O(1) + O(1) + O(1) + O(1)}
	
\EndFunction
\end{algorithmic}
\end{algorithm}

\begin{algorithm}
\caption{Implementaci\'on de crearPaquete}
\begin{algorithmic}[0]
\Function{iCrearPaquete}{\Inout{d}{estr\_dcnet}, \In{p}{paquete}}
	\State itPC $\gets$ Significado(d.porHostname, paquete.origen) \Comment{O(L)}
	\State itPaq $\gets$ AgregarRapido(SiguienteSignificado(itPC).paquetes, p) \Comment{O(copiar(nat) + O(L))} \Comment{O(L)}
	\State Encolar(SiguienteSignificado(itPC).cola, p.prioridad ,itPaq)	\Comment{O(log(n)), n cantidad de nodos)O(log(k)}
	\State Definir(SiguienteSignificado(itPC).paquetesPorID, IDpaquete, itPaq) \Comment{O(log(k))}
\EndFunction \Comment{O( L + log(k) )}
\end{algorithmic}
\end{algorithm}

\begin{algorithm}
\caption{Implementaci\'on de AvanzarSegundo}
\begin{algorithmic}[0]

\Function{iAvanzarSegundo}{inout d: estr\_dcnet}{}

	\State arreglo $\gets$ crearArreglo[$\#$Claves(d.computadoras)] de tupla(usado: bool, paquete: paquete, destino: string), donde paquete es tupla(IDpaquete: nat, prioridad: nat, origen: string, destino: string)
	\State \Comment {O(n) para calcular cantidad de claves, O(1) para crearlo}
	
	\For {(int i=0,  $<$ $\#$Claves(d.computadoras), i++)} \Comment {el ciclo se har\'a n veces}
	\State arreglo[i].usado = false \Comment {O(1)}
	\EndFor
	
	\comentario {Inicializo Iterador}
	\State itCompu $\gets$ crearIt(d.computadoras) \Comment {O(1)}
	\State i $\gets$ 0
	
	\State \Comment {Ciclo 1: Desencolo y guardo en arreglo auxiliar.}
	\While { (HaySiguiente(itCompu))} \Comment {el ciclo se har\'a a lo sumo n veces}
		\If {($\neg$(Vacia?(SiguienteSignificado(itCompu).cola)))} \Comment {O(1)}
			\comentario{Borro el de mayor priorirdad del heap:}
			\State itPaquete $\gets$ Desencolar(SiguienteSignificado(itCompu).cola)
			 \Comment {O(log k)}
			\comentario{Lo elimino del dicc AVL}
			\State Borrar(SiguienteSignificado(itCompu).paquetesPorID, Siguiente(itPaquete).IDpaquete)
			\State	\Comment {O(log k)}
			\comentario{Guardo el paquete en una variable}
			\State paqueteDesencolado $\gets$ Siguiente(itPaquete) \Comment {O(1)}
			\comentario{Lo elimino del conjunto lineal de paquetes}
			\State EliminarSiguiente(itPaquete) \Comment {O(1)}
			\comentario{Calculo proximo destino fijandome en la matriz}
			\comentario{El origen lo tengo en O(1) en el significado del iterador de compus.}
			\State origen $\gets$ (SiguienteSignificado(itCompu)).indice \Comment {O(1)}
			\comentario{El destino lo obtengo en O(L) buscando por hostname el destino del paquete, y luego guardo el indice.}
			\State itdestino $\gets$ Significado(d.porHostname, paqueteDesencolado.destino) 
				\Comment {O(L)}
			\State destino $\gets$ (SiguienteSignificado(itdestino)).indice 	\Comment {O(1)}
			\State proxDest $\gets$ d.caminos[origen][destino][1]	\Comment {O(1)}
			\comentario{Lo inserto en el arreglo junto con el destino s\'olo si el destino no era el final.}
			\If {(proxDest $\neq$ paqueteDesencolado.destino)}
				\State arreglo[i] $\gets$ $<$true, paqueteDesencolado, proxDest$>$ \Comment {O(1)}
			\EndIf
			\comentario{Aumento cantidad de env\'ios}
			\State SiguienteSignificado(itCompu).cantEnvios ++ \Comment {O(1)}
			\comentario{Actualizo conMasEnvios}
			\State envios $\gets$ SiguienteSignificado(itCompu).cantEnvios \Comment {O(1)}
			\If {(envios $>$ SiguienteSIgnificado(d.conMasEnvios).cantEnvios)} \Comment {O(1)}
				\State d.conMasEnvios $\gets$ itCompu
			\EndIf
		\EndIf
		\comentario{Avanzo de computadora}
		\State Avanzar(itCompu)	\Comment {O(1)}
		\State i++
	\EndWhile
	
	\algstore {ciclo2}
	
\end{algorithmic}
\end{algorithm}

\begin{algorithm}
\begin{algorithmic}[0]

	\algrestore {ciclo2}
	
	\State \Comment {Ciclo 2: Encolo los paquetes del vector a sus destinos correspondientes.}
	\State i $\gets$ 0
	\While { HaySiguiente(itCompu)} 	\Comment {el ciclo se har\'a a lo sumo n veces}
		\If {arreglo[i].usado}
			\comentario{Busco el proxDestino guardado en el arreglo por hostname.}
			\State itdestino $\gets$ Significado(d.porHostname, arreglo[i].destino) \Comment {O(L)}
			\comentario{Agrego el paquete al conjunto de paquetes del prox destino.}
			\State itpaquete $\gets$ AgregarRapido(SiguienteSignificado(itdestino).paquetes, arreglo[i].paquete) 
			\State \Comment {O(1)}
			\comentario{Encolo el heap del destino}
			\State prioridad $\gets$ (arreglo[i].paquete).prioridad
			\State Encolar(SiguienteSignificado(itdestino).cola, prioridad, itpaquete) \Comment {O(log k)}
			\comentario{Lo agrego en el dicc AVL.}
			\State IDpaq $\gets$ (arreglo[i].paquete).IDpaquete 		\Comment {O(1)}
			\State Definir(SiguienteSignificado(itdestino).paquetesPorID, IDpaq, itpaquete) \Comment {O(log k)}
		\EndIf
	\State i++
	\State Avanzar(itCompu)
	\EndWhile
	
\EndFunction \Comment{O( n * ( L + log(k) ) )}
\end{algorithmic}
\end{algorithm}

\begin{algorithm}
\caption{Implementaci\'on de PaqueteEnTransito?}
\begin{algorithmic}[0]
\Function{iPaqueteEnTransito?}{in d: estr\_dcnet, in p:IDpaquete}{$\rightarrow$ res: bool}
	\State res $\gets$ false \Comment{O(1)}
	\State itCompu $\gets$ crearIt(d.computadoras) \Comment{O(1)}
	\While { HaySiguiente(itCompu) $\&\&$ $\neg$res}  
		\newline  \Comment{a lo sumo n veces, la guarda es O(1)}
		\State itPaq $\gets$ crearIt(siguienteSignificado(itCompu).paquetes) \Comment{O(1)}
		\While {(HaySiguiente(itPaq) $\&\&$ Siguiente(itPaq).id $\neq$ p)} 
		\Comment{a lo sumo k veces, la guarda es O(1)}
			\State Avanzar(itPaq)  \Comment{O(1)}
		\EndWhile
		\If {Siguiente(itPaq) == p)} \Comment{O(1)}
			\State res $\gets$ True \Comment{O(1)}
		\EndIf
		\State Avanzar(itCompu) \Comment{O(1)}
	\EndWhile
\EndFunction \Comment{O(n * k)}
\end{algorithmic}
\end{algorithm}

\begin{algorithm}
\caption{Implementaci\'on de LaQueMasEnvi\'o}
\begin{algorithmic}[0]
\Function{iLaQueMasEnvi\'o}{in d: estr\_dcnet}{$\rightarrow$ res: hostname}
	\State res $\gets$ SiguienteClave(d.conMasEnvios)	
\EndFunction \Comment{O(1)}
\end{algorithmic}
\end{algorithm}



\begin{algorithm}
\caption{Implementaci\'on de ==}
\begin{algorithmic}[0]
\Function{iIgualdad}{in d1: estr\_dcnet, in d2: estr\_dcnet}{$\rightarrow$ res: bool}
	\comentario {Comparo redes usando == de red}
	\State res $\gets$ (d1.red == d2.red)						\Comment {O(n*n * ( L + n*n + m ) + n*m*m)}
	\If {(res)} 													\Comment {O(1)}
		\State itCompu $\gets$ crearIt(d1.computadoras)  		\Comment {O(1)}
		\State string host 										\Comment {O(1)}
		\comentario {Recorro las computadoras}
		\While { (HaySiguiente(itCompu) $\&\&$ res) }	\Comment {itero O(n) veces, la guarda es O(1)}
			\State host $\gets$ SiguienteClave(itCompu)			 \Comment {O(1)}
			\comentario{Comparo enEspera usando == de conjunto lineal, y cant. enviados}
			\State res $\gets$ (enEspera(d1, host) == enEspera(d2, host) $\&\&$ \newline  cantidadEnviados(d1,host) == cantidadEnviados(d2,host))  \Comment {O(L)}
			\State itpaq $\gets$ crearIt(SiguienteSignificado(itCompu).paquetes) \Comment {O(1)}
			\State int j $\gets$ 0												\Comment {O(1)}
			\State nat id														\Comment {O(1)}
			\comentario{Recorro paquetes de cada computadora}
			\While { (HaySiguiente(itpaq) $\&\&$ res )}	\Comment {itero O(k) veces, la guarda es O(1)}
				\State id $\gets$ Siguiente(itpaq).IDpaquete 					\Comment {O(1)}
				\comentario {Comparo caminosRecorridos usando == de listas enlazadas}
				\State res $\gets$ (caminoRecorrido(d1, id) == caminoRecorrido(d2, id)) \Comment {O(n * log(k))}
				\State avanzar(itpaq)											\Comment {O(1)}
			\EndWhile
		\State avanzar (itCompu)													\Comment {O(1)}
		\EndWhile
	\EndIf
\EndFunction \Comment{O(n*n * ( L + n*n + m + log(k) ) + n*(m*m + L))}
\end{algorithmic}
\end{algorithm}

\end{Algoritmos}

\clearpage


