\begin{Interfaz}
  
  %\textbf{usa}: \tadNombre{}.
  
  \textbf{se explica con}: \tadNombre{AB($\sigma$)}.
  
  \textbf{g\'eneros}: \TipoVariable{ab($\sigma$)}.

  \Titulos{Operaciones}

  \InterfazFuncion{nil?}{\In{a}{ab($\sigma$)}}{Bool}
  {res $\igobs$ nil?(a)}
  [O(1)]
  [Indica si el arbol est\'a vac\'io.]

  \InterfazFuncion{raiz}{\In{a}{ab($\sigma$)}}{$\sigma$}
  [$\neg$nil?(a)]
  {res $\igobs$ raiz(a)}
  [O(1)]
  [Devulve el valor de la raiz del arbol.]
  [El valor se devuelve por referencia, res es modificable.]
  
  \InterfazFuncion{izq}{\In{a}{ab($\sigma$}}{ab($\sigma$)}
  [$\neg$nil?(a)]  
  {res $\igobs$ vacio()}
  [O(1)]
  [Devulve el subárbol izquierdo.]
  [El subárbol se devuelve por referencia, res es modificable.]
  
  \InterfazFuncion{setearIzq}{\Inout{a}{ab($\sigma$}, \In{i}{ab($\sigma$}}{}
  [$\neg$nil?(a)]  
  {res $\igobs$ bin(i, raiz(a), der(a))}
  [O(1)]
  [Asigna i como nuevo subárbol izquierdo.]
  []
  
  \InterfazFuncion{der}{\In{a}{ab($\sigma$}}{ab($\sigma$)}
  [$\neg$nil?(a)]  
  {res $\igobs$ vacio()}
  [O(1)]
  [Devulve el subárbol derecho.]
  [El subárbol se devuelve por referencia, res es modificable.]
  
  \InterfazFuncion{setearDer}{\Inout{a}{ab($\sigma$}, \In{d}{ab($\sigma$}}{}
  [$\neg$nil?(a)]  
  {res $\igobs$ bin(izq(a), raiz(a), d)}
  [O(1)]
  [Asigna d como nuevo subárbol derecho.]
  []
  
  \InterfazFuncion{nil}{}{ab($\sigma$)}
  []
  {res $\igobs$ nil()}
  [O(1)]
  [Devuelve un \'arbol binario vacio.]
  
  \InterfazFuncion{bin}{\In{i}{ab($\sigma$)}, \In{a}{$\sigma$}, \In{d}{ab($\sigma$)}}{ab($\sigma$)}
  []
  {res $\igobs$ bin(i,a,d)}
  [O(1)]
  [Crea un \'arbol binario con raiz a, hijo derecho d e hijo izquierdo i.]

\end{Interfaz}

\newpage 

\begin{Representacion}

\begin{Estructura}{ab($\sigma$)}[puntero(nodo($\sigma$))]
   
\begin{Tupla}[nodo($\alpha$)]
	\tupItem{izq}{ab($\sigma$)} \newline \nomoreitems
	\tupItem{der}{ab($\sigma$)} \newline \nomoreitems
	\tupItem{valor}{$\sigma$} \newline \nomoreitems
\end{Tupla} 

\end{Estructura}

\comentario{Rep en castellano:}
\comentario{1: Si un elemento es el hijo (izquierdo o derecho) de otro, entonces no es el hijo de ning\'un otro}
\comentario{2: Si un elemento es hijo izquierdo de cierto elemento, no puede ser tambi\'en el derecho}

Rep: AB($\alpha, \sigma$) $\rightarrow$ bool
Rep($estr$) $\equiv$ true $\iff$
\comentario{1} $((\forall n_1, n_2, n_3 : nodo(\alpha, \sigma)) ((n_1 \in arbol(estr.primero) \wedge\ n_2 \in arbol(estr.primero) \wedge\ n_3 \in arbol(estr.primero) \wedge\ (n_1 = n_2\rightarrow izq \vee\ n_1 = n_2\rightarrow der) \wedge\ n_2 \neq n_3) \Rightarrow_L\ (n_1 \neq n_3\rightarrow izq \wedge n_1 \neq n_3\rightarrow der))\ \wedge$
\comentario{2} $((\forall n_1, n_2 : nodo(\alpha, \sigma)) ((n_1 \in arbol(estr.primero) \wedge\ n_2 \in arbol(estr.primero) \wedge n_1 = n_2\rightarrow izq) \Rightarrow_L\ n_1 \neq n_2\rightarrow der)$

\textbf{}
\textbf{}

\textbf{Especificaci\'on de las operaciones auxiliares utilizadas para Rep y Abs}
\tadOperacion{arbol}{nodo($\alpha$, $\sigma$)}{conj(nodo($\alpha$, $\sigma$))}{}
\tadOperacion{caminoHastaRaiz}{nodo($\alpha$, $\sigma$)}{nat}{}
\tadAxioma{arbol(n)}{
\IF {$n.izq \neq null \wedge n.der \neq null$}
THEN {Ag($n,arbol(n.izq)\cup arbol(n.der)$)}
ELSE { \IF {$n.izq \neq null$}
		THEN {Ag($n,arbol(n.izq)$)}
		ELSE { \LIF\ $n.der \neq null$ \LTHEN\ Ag($n,arbol(n.der)$) \LELSE\ Ag($n,\emptyset$) \LFI}
		FI}
FI}

\tadAxioma{caminoHastaRaiz(n)}{\textbf{if} $n.padre = null$ \textbf{then} $0$ \textbf{else} $caminoHastaRaiz(n.padre)+1$ \textbf{fi}}

\end{Representacion}

\begin{Algoritmos}

\begin{algorithm}
\caption{Implementaci\'on de nil?}
\begin{algorithmic}[0]
\Function{iNil?}{in a: ab($\sigma$)}{$\rightarrow$ res: bool}
	\State res $\gets$ a == NULO)
\EndFunction
\end{algorithmic}
\end{algorithm}

\begin{algorithm}
\caption{Implementaci\'on de raiz}
\begin{algorithmic}[0]
\Function{iRaiz}{in a: ab($\sigma$)}{$\rightarrow$ res: $\sigma$}
	\State res $\gets$ a->raiz
\EndFunction
\end{algorithmic}
\end{algorithm}

\begin{algorithm}
\caption{Implementaci\'on de izq}
\begin{algorithmic}[0]
\Function{iIzq}{in a: ab($\sigma$)}{$\rightarrow$ res: ab($\sigma$)}
	\State res $\gets$ a->izq
\EndFunction
\end{algorithmic}
\end{algorithm}

\begin{algorithm}
\caption{Implementaci\'on de setearIzq}
\begin{algorithmic}[0]
\Function{iSetearIzq}{inout a: ab($\sigma$), in i:ab($\sigma$)}{}
	\State a->izq $\gets$ i
\EndFunction
\end{algorithmic}
\end{algorithm}

\begin{algorithm}
\caption{Implementaci\'on de der}
\begin{algorithmic}[0]
\Function{iDer}{in a: ab($\sigma$)}{$\rightarrow$ res: ab($\sigma$)}
	\State res $\gets$ a->der
\EndFunction
\end{algorithmic}
\end{algorithm}

\begin{algorithm}
\caption{Implementaci\'on de setearDer}
\begin{algorithmic}[0]
\Function{iSetearDer}{inout a: ab($\sigma$), in d:ab($\sigma$)}{}
	\State a->der $\gets$ d
\EndFunction
\end{algorithmic}
\end{algorithm}

\begin{algorithm}
\caption{Implementaci\'on de nil}
\begin{algorithmic}[0]
\Function{iNil}{}{$\rightarrow$ res: ab($\sigma$)}
	\State res $\gets$ NULL
\EndFunction
\end{algorithmic}
\end{algorithm}

\begin{algorithm}
\caption{Implementaci\'on de bin}
\begin{algorithmic}[0]
\Function{iIzq}{in i: ab($\sigma$), in a: $\sigma$, in d: ab($\sigma$)}{$\rightarrow$ res: ab($\sigma$)}
	\State res->izq $\gets$ i
	\State res->der $\gets$ d
	\State res->raiz $\gets$ a
\EndFunction
\end{algorithmic}
\end{algorithm}

\end{Algoritmos}