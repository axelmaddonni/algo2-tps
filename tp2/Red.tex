\begin{Interfaz}

  \textbf{usa}: \tadNombre{conj($\alpha$), itConj($\alpha$), lista($\alpha$), itLista($\alpha$)}.
  
  \textbf{se explica con}: \tadNombre{Red}.

  \textbf{g\'eneros}: \TipoVariable{red}.

  \Titulos{Operaciones de Red}

  \InterfazFuncion{Computadoras}{\In{r}{red}}{conj(hostname)}
  {$res$ $=_{obs}$ dameHostnames(computadoras($r$))}%
  [O(n)]
  [devuelve el conjunto de las computadoras.]
  [res se devuelve por copia.]

  \InterfazFuncion{Conectadas?}{\In{r}{red}, \In{c1}{hostname}, \In{c2}{hostname}}{bool}
  [$c1,\ c2$ $\in$ dameHostnames(computadoras($r$))]
  {$res$ $=_{obs}$ conectadas?($r$, dameCompu($c1$), dameCompu($c2$))}
  [O(n*L)]
  [indica si las computadoras estan conectadas por alguna de sus interfaces.]
  []

  \InterfazFuncion{InterfazUsada}{\In{r}{red}, \In{c1}{hostname}, \In{c2}{hostname}}{interfaz}
  [conectadas?($r$, dameCompu($c1$), dameCompu($c2$))]
  {$res$ $=_{obs}$ interfazUsada($r$, dameCompu($c1$), dameCompu($c2$))}
  [O(n*L)]
  [devuelve la interfaz por la cual estan conectadas c1 y c2.]
  []

  \InterfazFuncion{IniciarRed}{}{red}
  [true]
  {$res$ $=_{obs}$ iniciarRed()}
  [O(1)]
  [crear una nueva Red.]
  []
  
  \InterfazFuncion{AgregarCompu}{\Inout{r}{red}, \In{c1}{compu}}{}
  [$r$ = $r_0$ $\wedge$
  (\paratodo{compu}{c}) $c$ $\in$ computadoras($r_0$) $\Rightarrow$ ip($c$) $\neq$ $c1$]
  {$r$ $=_{obs}$ agregarComputadora($r_0$, $c1$)}
  [O(L+i) i=cantidad de interfaces]
  [agregar una computadora a la Red.]
  [la computadora se agrega por copia.]

  \InterfazFuncion{Conectar}{\In{r}{red}, \In{c_1}{hostname}, \In{i_1}{interfaz} \In{c_2}{hostname}, \In{i_2}{interfaz}}{}
  [$r$ = $r_0$ $\wedge$ $c_1, c_2$ $\in$ dameHostnames(computadoras($r$)) $\wedge$ $c_1$ $\neq$ $c_2$	$\wedge$ \newline $\neg$conectadas?($r$, dameCompu($c_1$), dameCompu($c_2$)) $\wedge$ $\neg$usaInterfaz?($r$, dameCompu($c_1$), $i_1$) $\wedge$ $\neg$usaInterfaz?($r$, dameCompu($c_2$), $i_2$) $\wedge$ $i_1$ $\in$ dameCompu($c_1$).interfaces $\wedge$ $i_2$ $\in$ dameCompu($c_2$).interfaces]
  {$r$ $=_{obs}$ (conectar($r_0$, dameCompu($c_1$), $i_1$, dameCompu($c_2$), $i_2$)}
  []
  [conectar dos computadoras de la red.]
  []
   
  \InterfazFuncion{Vecinos}{\In{r}{red}, \In{c}{hostname}}{conj(hostname)}
  [$c$ $\in$ dameHostnames(computadoras($r$))]
  {$res$ $=_{obs}$ dameHostnames(vecinos($r$, dameCompu($c$)))}
  [O(n*L)]
  [da el conjunto de computadoras vecinas.]
  [el conjunto se devuelve por copia.]

  \InterfazFuncion{UsaInterfaz?}{\In{r}{red}, \In{c}{hostname}, \In{i}{interfaz}}{bool}
  [$c$ $\in$ dameHostnames(computadoras($r$))]
  {$res$ $=_{obs}$ usaInterfaz?($r$, dameCompu($c$), $i$)}
  [O(n)]
  [indica si la interfaz est\'a siendo utilizada.]
  []

\newpage

  \InterfazFuncion{CaminosMinimos}{\In{r}{red}, \In{c_1}{hostname}, \In{c_2}{hostname}}{conj(lista(hostname))}
  [$c1,\ c2$ $\in$ dameHostnames(computadoras($r$))]
  {$res$ $=_{obs}$ dameCaminosdeHostnames(caminosMinimos($r$, dameCompu($c_1$), dameCompu($c_2$)))}
  [O(n*L)]
  [devuelve los conjuntos de caminos minimos entre las computadoras ingresadas.]
  [res se devuelve por copia.]
  
  \InterfazFuncion{HayCamino?}{\In{r}{red}, \In{c_1}{hostname}, \In{c_2}{hostname}}{bool}
  [$c1,\ c2$ $\in$ dameHostnames(computadoras($r$))]
  {$res$ $=_{obs}$ hayCamino?($r$, dameCompu($c_1$), dameCompu($c_2$))}
  [O(n*n)]
  [indica si las computadoras son alcanzables mediante alg\'un camino.]
  []
  
  \InterfazFuncion{$\bullet == \bullet$}{\In{r_1}{red}, \In{r_2}{red}}{bool}
  [true]
  {$res$ =$_{obs}$ ($r_1$ =$_{obs}$ $r_2$}%
  [O(n*n * ( L + n*n + m ) + n*m*m)]
  [indica si dos redes son iguales.]
  [] 
  
   \textbf{donde:} \newline 
   \TipoVariable{hostname} es \TipoVariable{string}, \newline
   \TipoVariable{interfaz} es \TipoVariable{nat}, \newline
   \TipoVariable{compu} es \TipoVariable{tupla}< ip: \TipoVariable{hostname}, interfaces: \TipoVariable{conj}(\TipoVariable{interfaz})>.
   
\end{Interfaz}

\textbf{} %dejo un espacio

\textbf{Especificaci\'on de las operaciones auxiliares utilizadas en la interfaz (no exportadas)}

\begin{tad}{\tadNombre{Red extendida}}

\tadExtiende{\tadNombre{Red}}

\setlength{\tadAnchoTipoFunciones}{0pt}

\tadOtrasOperaciones 

\tadOperacion{damehostnames}{conj(compu)}{conj(hostname)}{}
\tadOperacion{dameCompu}{red /r, hostname /s}{compu}{$s$ $\in$ hostnames($r$)}
\tadOperacion{auxDameCompu}{red /r, hostname /s, conj(compu) /cc}{compu}{$s$ $\in$ hostnames($r$) $\wedge$ $cc$ $\subset$ computadoras($r$)}
\tadOperacion{dameCaminosDeHostnames}{conj(secu(compu))}{conj(secu(hostname))}{}
\tadOperacion{dameSecuDeHostnames}{secu(compu)}{secu(hostname)}{}

\tadAxiomas[\paratodo{red}{r}, \paratodo{conj(compu)}{cc}, \paratodo{hostname}{s}, \paratodo{conj(secu(compu))}{cs}, \paratodo{secu(compu)}{secu}]

\tadAxioma{dameHostnames($cc$)}
{ \IF {vacio?($cc$)} 
THEN {$\emptyset$}
ELSE {Ag( ip(dameUno($cc$)), dameHostnames(sinUno($cc$)) )}
FI}

\tadAxioma{dameCompu($r$, $s$)}{auxDameCompu($r$, $s$, computadoras($r$))}

\tadAxioma{auxDameCompu($r$, $s$, $cc$)}
{ \IF {ip(dameUno($cc$)) = $s$} 
THEN {dameUno($cc$)}
ELSE {auxDameCompu($r$, $s$, sinUno($cc$))}
FI}

\tadAxioma{dameCaminosDeHostnames($cs$)}
{ \IF {vacio?($cs$)} 
THEN {$\emptyset$}
ELSE {Ag( dameSecuDeHostnames(dameUno($cs$)), dameCaminosDeHostnames(sinUno($cs$)) )}
FI}

\tadAxioma{dameSecuDeHostnames($secu$)}
{ \IF {vacia?($secu$)} 
THEN {$<>$}
ELSE {ip(prim($secu$)) $\bullet$ dameSecuDeHostnames(fin($secu$)) }
FI}

\end{tad}

\begin{Representacion}

\begin{Estructura}{red}[estr\_red]

\textbf{donde:} \newline 
\TipoVariable{estr\_red} es \TipoVariable{dicc(hostname, datos)} \newline
   
\begin{Tupla}[datos]
\tupItem{interfaces}{\TipoVariable{conj(interfaz)}} \newline \nomoreitems{}	 
\tupItem{conexiones}{\TipoVariable{dicc(interfaz, hostname)}} \newline \nomoreitems{}
\tupItem{alcanzables}{\TipoVariable{dicc(dest: hostname, caminos: conj(lista(hostname)))}}
\end{Tupla}

\TipoVariable{hostname} es \TipoVariable{string}, \TipoVariable{interfaz} es \TipoVariable{nat}.
      
\end{Estructura}

\tadAlinearFunciones{REP}{estr\_red /e}

\Rep[estr\_red]{
	($\forall$ $c$: hostname, $c$ $\in$ claves($e$)) \textcolor{red}{(} 
	\comentario{1} claves(obtener($e$, $c$).conexiones) $\subseteq$ obtener($e$, $c$).interfaces $\wedge$ \newline
	 ($\forall$ $i$: interfaz, $i$ $\in$ claves(obtener($e$, $c$).conexiones)) \textcolor{blue}{(}  
	\comentario{2} obtener(obtener($e$, $c$).conexiones, $i$) $\in$ claves($e$) $\wedge$ 	
	\comentario{3} obtener(obtener($e$, $c$).conexiones, $i$) $\neq$ $c$ 	$\wedge$ 
	\comentario{4} ($\neg$ $\exists$ $i\sp{\prime}$: interfaz, $i\sp{\prime}$ $\in$ claves(obtener($e$, $c$).conexiones), $i$ $\neq$ $i\sp{\prime}$) obtener(obtener($e$, $c$).conexiones, $i$) == obtener(obtener($e$, $c$).conexiones, $i\sp{\prime}$) $\wedge$ 
	\comentario{5} ($\forall$ $h$: hostname) ( $h$ == obtener(obtener($e$, $c$).conexiones, $i$)
	$\Rightarrow$ ($\exists$ $i\sp{\prime}$:int) obtener(obtener($e$, $h$).conexiones, $i\sp{\prime}$) == $c$ ) 
	\textcolor{blue}{)} $\wedge$ 
	\comentario{6} claves(obtener($e$, $c$).alcanzables $\subseteq$ claves($e$) $\wedge$ \newline
	 ($\forall$ $a$: hostname, $a$ $\in$ claves(obtener($e$, $c$).alcanzables) \textcolor{blue}{(} 
	\comentario{7} $a$ $\neq$ $c$ $\wedge$ 
	\comentario{8} ($\exists$ $s$: secu(hostname)) esCaminoV\'alido($c$, $a$, $s$)  $\wedge$ 
	\comentario{9} $\#$obtener(obtener($e$, $c$).alcanzables, $a$) > 0 $\wedge_L$ \newline
	($\forall$ $camino$: secu(hostname), $camino$ $\in$ obtener(obtener($e$, $c$).alcanzables, $a$) (
	\comentario{10} esCaminoV\'alido($c$, $a$, $camino$) $\wedge$ 
	\comentario{11} $\neg$($\exists$ $camino\sp{\prime}$: secu(hostname), $camino$ $\neq$ $camino\sp{\prime}$, esCaminoV\'alido($c$, $a$, $camino\sp{\prime}$)) long($camino\sp{\prime}$) < long($camino$ ) $\wedge$ 
	\comentario{12} $\neg$($\exists$ $camino\sp{\prime}$: secu(hostname), $camino$ $\neq$ $camino\sp{\prime}$, esCaminoV\'alido($c$, $a$, $camino\sp{\prime}$), long($camino$ == long($camino\sp{\prime}$)) ($camino\sp{\prime}$ $\notin$ obtener(obtener($e$, $c$).alcanzables, $a$) )
	\newline \textcolor{blue}{)} \textcolor{red}{)}
}

La abreviatura $esCaminoValido$ usada en el Rep se debe leer: (no son funciones, son abreviaturas para hacer m\'as f\'acil la lectura)

\tadAxioma{$esCaminoValido$ ($orig$, $dest$, $secu$)} {  
	( prim($secu$) == $orig$ $\wedge$ \newline
	($\forall$ $i$: nat, 0 $<$ $i$ $<$ long($secu$) ) esVecino ($secu[i]$, $secu[i+1]$) $\wedge$ \newline
	$secu$ [long($secu$)-1] == $dest$ $\wedge$ \newline
	sinRepetidos($secu$) )
	}

Con $esVecino$ ($h1$, $h2$) $\equiv$  ($\exists$ $i$: interfaz) $h2$ == obtener (obtener($e$, $h1$).conexiones, $i$) 

\newpage

\comentario{Rep en Castellano: \newline
Para cada computadora: \newline
1: Las interfaces usadas pertenecen al conjunto de interfaces de la compu.\newline
2: Los vecinos perteneces a las computadoras de la red.\newline
3: Los vecinos son distintos a la compu actual.\newline
4: Los vecinos no se repiten.\newline
5: Las conexiones son bidireccionales.\newline
6: Los alcanzables pertenecen a las computadoras de la red.\newline
7: Los alcanzables son distintos a la actual.\newline
8: Los alcanzables tienen un camino v\'alido hacia ellos desde la actual.\newline
9: Para cada alcanzable, el conjunto de camiinos v\'alidos no es vac\'io.\newline
10: Todos los caminos en el diccionario alcanzables son v\'alidos.\newline
11: Los caminos son m\'inimos.\newline
12: Est\'an todos los m\'inimos.
}

\AbsFc[estr\_red]{red}[e]{$r$ $|$ computadoras ($r$) = dameComputadoras($e$) $\wedge_L$ \newline
	($\forall$ $c1$, $c2$: compu, $c1$, $c2$ $\in$ computadoras($r$)) conectadas?($r$, $c1$, $c2$) = ($\exists$ $i$: interfaz) ( $c2$.ip = obtener(obtener($e$, $c1$.ip).conexiones, $i$) ) $\wedge_L$ \newline
	interfazUsada($r$, $c1$, $c2$) = buscarClave (obtener($e$, $c1$.ip).conexiones, $c2$.ip)	
	}

\textbf{} %dejo un espacio

\textbf{Especificaci\'on de las funciones auxiliares utilizadas en abs}

\tadAlinearFunciones{auxDameComputadoras}{dicc(interfaz, hostname), hostname, conj(interfaz)}

\tadOperacion{dameComputadoras}{dicc(hostname; X)}{conj(computadoras)}{}
\tadOperacion{auxDameComputadoras}{dicc(hostname; X), conj(hostname)}{conj(computadoras)}{}
\tadOperacion{buscarClave}{dicc(interfaz; hostname), hostname}{interfaz}{}
\tadOperacion{auxBuscarClave}{dicc(interfaz; hostname), hostname, conj(interfaz)}{interfaz}{}

\tadAxiomas[\paratodo{dicc(hostname, X)}{e}, \paratodo{dicc(interfaz, hostname)}{d}, \paratodo{conj(hostname)}{cc}, \paratodo{conj(interfaz)}{ci}, \paratodo{hostname}{h}]

\tadAxioma{dameComputadoras($e$)}{auxDameComputadoras($e$, claves($e$)}
\tadAxioma{auxDameComputadoras($e$, $cc$)}
{ \IF {$\emptyset$?($cc$)} 
THEN {$\emptyset$}
ELSE {Ag( <dameUno($cc$), obtener($e$, dameUno($cc$)).interfaces>, auxDameComputadoras($e$, sinUno($cc$)) )}
FI}

\tadAxioma{buscarClave($d$, $h$)}{auxBuscarClave($d$, $h$, claves($d$))}
\tadAxioma{auxBuscarClave($d$, $h$, $ci$)}
{ \IF {obtener($d$, dameUno($cc$)) = $h$} 
THEN {dameUno($cc$)}
ELSE {auxBuscarClave($d$, $h$, sinUno($ci$))}
FI}

\end{Representacion}

\newpage

\begin{Algoritmos}
\begin{algorithm}
\caption{Implementaci\'on de Computadoras}
\begin{algorithmic}[0]
\Function{iComputadoras}{in r: estr\_red}{$\rightarrow$ res: conj(hostname)}
	\State it $\gets$ crearIt(r) \Comment{O(1)}
	\State res $\gets$ Vacio() \Comment{conjunto} \Comment{O(1)}
	\While{HaySiguiente(it)} \Comment{Guarda: O(1)} \Comment{El ciclo se ejecuta n veces} \Comment{O(n)}
		\State Agregar(res, SiguienteClave(it)) \Comment{O(1)}
		\State Avanzar(it) \Comment{O(1)}
	\EndWhile
\EndFunction \Comment{\textbf{O(n)}}
\end{algorithmic}
\end{algorithm}

\begin{algorithm}
\caption{Implementaci\'on de Conectadas?}
\begin{algorithmic}[0]
\Function{iConectadas?}{in r: estr\_red, in c1: hostname, in c2: hostname}{$\rightarrow$ res: bool}
	\State it $\gets$ CrearIt(Significado(r,c1).conexiones) \Comment{O(n)}
	\State res $\gets$ FALSE \Comment{O(1)}
	\While{HaySiguiente(it) $\&\&$ $\neg$res} \Comment{Guarda: O(1)} \Comment{El ciclo se ejecuta a lo sumo n-1 veces} \Comment{O(n)}
		\If{SiguienteClave(it)==c2} \Comment{O(L)}
			\State res $\gets$ TRUE
		\EndIf
		\State Avanzar(it) \Comment{O(1)}
	\EndWhile
\EndFunction \Comment{\textbf{O(n*L)}}
\end{algorithmic}
\end{algorithm}

\begin{algorithm}
\caption{Implementaci\'on de InterfazUsada}
\begin{algorithmic}[0]
\Function{iInterfazUsada}{in r: estr\_red, in c1: hostname, in c2: hostname}{$\rightarrow$ res: interfaz}
	\State it $\gets$ CrearIt(significado(r,c1).conexiones) \Comment{O(n)}
	\While{HaySiguiente(it)} \Comment{Guarda: O(1)} \Comment{El ciclo se ejecuta a lo sumo n-1 veces} \Comment{O(n)}
		\If{SiguienteSignificado(it)==c2} \Comment{O(L)}
			\State res $\gets$ SiguienteClave(it) \Comment{nat por copia} \Comment{O(copiar(nat))}
		\EndIf
		\State Avanzar(it) \Comment{O(1)}
	\EndWhile
\EndFunction \Comment{\textbf{O(n*L)}}
\end{algorithmic}
\end{algorithm}

\begin{algorithm}
\caption{Implementaci\'on de IniciarRed}
\begin{algorithmic}[0]
\Function{iIniciarRed()}{}{$\rightarrow$ res: estr\_red}
	\State res $\gets$ Vacio() \Comment{Diccionario} \Comment{O(1)}
\EndFunction \Comment{\textbf{O(1)}}
\end{algorithmic}
\end{algorithm}

\begin{algorithm}
\caption{Implementaci\'on de AgregarCompu}
\begin{algorithmic}[0]
\Function{iAgregarCompu}{inout r: estr\_red, in c1: compu}{}
	\State nuevoDiccVacio $\gets$ Vacio() \Comment{Diccionario} \Comment{O(1)}
	\State DefinirRapido(r, c1.ip, tupla(Copiar(c1.interfaces), nuevoDiccVacio, nuevoDiccVacio)) \Comment{O(Copiar(sL) + Copiar(conj(interfaz)) con sL=string de largo L}
\EndFunction \Comment{\textbf{O(L + i) con i=cantidad de interfaces}}
\end{algorithmic}
\end{algorithm}

\begin{algorithm}
\caption{Implementaci\'on de Conectar}
\begin{algorithmic}[0]
\Function{iConectar}{inout r: estr\_red, in c1: hostname}{}

\EndFunction
\end{algorithmic}
\end{algorithm}

\begin{algorithm}
\caption{Implementaci\'on de Vecinos}
\begin{algorithmic}[0]
\Function{iVecinos}{inout r: estr\_red, in c1: hostname}{$\rightarrow$ res: conj(hostname)}
	\State it $\gets$ CrearIt(Significado(r,c1).conexiones) \Comment{O(1) + O(n)}
	\State res $\gets$ Vacio() \Comment{Conjunto} \Comment{O(1)}
	\While{HaySiguiente(it)} \Comment{Guarda: O(1)} \Comment{El ciclo se ejecuta a lo sumo n-1 veces} \Comment{O(n)}
		\State AgregarRapido(res, SiguienteSignificado(it)) \Comment{O(L)}
		\State Avanzar(it) \Comment{O(1)}
	\EndWhile
\EndFunction \Comment{\textbf{O(n*L)}}
\end{algorithmic}
\end{algorithm}

\begin{algorithm}
\caption{Implementaci\'on de UsaInterfaz}
\begin{algorithmic}[0]
\Function{iUsaInterfaz}{in r: estr\_red, in c: hostname, in i: interfaz}{$\rightarrow$ res: bool}
	\State res $\gets$ Definido?(Significado(r,c).conexiones,i) \Comment{O(comparar(nat)*n)}
\EndFunction \Comment{\textbf{O(n)}}
\end{algorithmic}
\end{algorithm}

\begin{algorithm}
\caption{Implementaci\'on de CaminosMinimos}
\begin{algorithmic}[0]
\Function{iCaminosMinimos}{in r: estr\_red, in c1: hostname, in c2: hostname}{$\rightarrow$ res: conj(lista(hostname))}
	\State itCaminos $\gets$ crearIt(Significado(Significado(r,c1).alcanzables, c2)) \Comment{O(1) + O(L*n)}
	\State res $\gets$ Vacio() \Comment{Conjunto} \Comment{O(1)}
	\While{HaySiguiente(itCaminos)}
		\State AgregarRapido(res, Siguiente(itCaminos)) \Comment{O(1)}
		\State Avanzar(itCaminos) \Comment{O(1)}	
	\EndWhile
\EndFunction \Comment{\textbf{O(n*L)}}
\end{algorithmic}
\end{algorithm}

\begin{algorithm}
\caption{Implementaci\'on de HayCamino?}
\begin{algorithmic}[0]
\Function{iHayCamino?}{in r: estr\_red, in c1: hostname, in c2: hostname}{$\rightarrow$ res: bool}
	\State res $\gets$ Definido?(Significado(r,c1).alcanzables, c2) \Comment{O(n*n)}
\EndFunction \Comment{\textbf{O(n*n)}}
\end{algorithmic}
\end{algorithm}

\begin{algorithm}
\caption{Implementaci\'on de ==}
\begin{algorithmic}[0]
\Function{iIgualdad}{in r1: estr\_red, in r2: estr\_red}{$\rightarrow$ res: bool}
	\State res $\gets$ TRUE	\Comment{O(1)}
	\If{$\neg$($\#$Claves(r1)==$\#$Claves(r2))} \Comment{O(comparar(nat))}
\comentario{Si la cantidad de claves son distintas => las redes son distintas}
		\State res $\gets$ FALSE \Comment{O(1)}
	\Else
		\State itRed1 $\gets$ CrearIt(r1) \Comment{O(1)}
		\While{HaySiguiente(itRed1) $\&\&$ res} \Comment{Guarda: O(1)} \Comment{Se ejecuta n veces} \Comment{O(n)}
\comentario{Recorro la red 1 y me fijo para cada una de sus computadoras}
			\If{$\neg$(Definido?(r2, SiguienteClave(itRed1))} \Comment{O(L*n)}
\comentario{Si no está definido su hostname en la red 2 => las redes son distintas}
				\State res $\gets$ FALSE \Comment{O(1)}
			\Else
				\State Compu2 $\gets$ Significado(r2,SiguienteClave(itRed1)) \Comment{O(L*n)}
				\State Compu1 $\gets$ SiguienteSignificado(itRed1) \Comment{O(1)}
\comentario{Tomo las computadoras de red 1 y red 2 con el mismo hostname y las comparo}
				\If{$\neg$(Comp1.interfaces == Comp2.interfaces)} \Comment{O(m*m) con m=cantidad de interfaces}
\comentario{Si sus interfaces son distintas => las redes son distintas}
					\State res $\gets$ FALSE \Comment{O(1)}
				\EndIf
				
				\If{$\neg$(Comp1.conexiones == Comp2.conexiones)}
\comentario{Si sus conexiones son distintas => las redes son distintas}
					\State res $\gets$ FALSE \Comment{O(1)}
				\EndIf
				\If{$\neg$($\#$Claves(Compu1.alcanzables)==$\#$Claves(Compu2.alcanzables))}
\comentario{Si sus cantidades de alcanzables son distintas => las redes son distintas}
					\State res $\gets$ FALSE \Comment{O(1)}
				\Else
					\State itAlc1 $\gets$ CrearIt(Compu1.alcanzables) \Comment{O(1)}
					\While{HaySiguiente(itAlc1) $\&\&$ res} \Comment{se ejecuta a lo sumo n$-$1 veces} \Comment{O(n)}
\comentario{Para cada alcanzable de la computadora de la red 1}
						\If{$\neg$(Definido?(Comp2.alcanzables, SiguienteClave(itAlc1)))} \Comment{O(m)}
\comentario{Si no est\'a definida en los alcanzables de la compu de la red 2 => las redes son distintas}
							\State res $\gets$ FALSE
						\Else
							\State Caminos1 $\gets$ SiguienteSignificado(itAlc1) \Comment{O(1)}
							\State Caminos2 $\gets$ Significado(Comp2.alcanzables, itAlc1) \Comment{O(n)}
\comentario{Me guardo los 2 conjuntos de caminos (de la compu de la red 1 y la de la red 2)}
							\If{$\neg$(Longitud(Caminos1) == Longitud(Caminos2))} \Comment{O(comparar(nat))}
\comentario{Si sus cantidades son distintas => las redes son distintas}
								\State res $\gets$ FALSE
							\Else
								\State itCaminos1 $\gets$ CrearIt(Caminos1) \Comment{O(1)}
								\While{HaySiguiente(itCaminos1) $\&\&$ res}
\comentario{Para cada camino en el conjunto de caminos de la compu de la red 1}
\comentario{Recorro los caminos de la compu de la red 2}
									\State itCaminos2 $\gets$ CrearIt(Caminos2) \Comment{O(1)}
									\State noEncontro $\gets$ TRUE \Comment{O(1)}
									\While{HaySiguiente(itCaminos2) $\&\&$ noEncontro}
\comentario{Busco que el camino de la compu de la red 1 est\'e en la compu de la red 2}
										\If{Siguiente(itCaminos2) == Siguiente(ItCaminos1)}
											\State noEncontro $\gets$ FALSE
										\EndIf
										\State Avanzar(itCaminos2) \Comment{O(1)}
									\EndWhile
									\If{noEncontro} \Comment{O(1)}
\comentario{Si no encontr\'o alguno => las redes son distintas}
										\State res $\gets$ FALSE \Comment{O(1)}
									\EndIf
									\State Avanzar(itCaminos1) \Comment{O(1)}
								\EndWhile
							\EndIf
						\EndIf
						\State Avanzar(itAlc1) \Comment{O(1)}
					\EndWhile
				\EndIf
			\EndIf
			\State Avanzar(itRed1) \Comment{O(1)}
		\EndWhile
	\EndIf
\EndFunction \Comment{\textbf{O(n*n * ( L + n*n + m ) + n*m*m)}}
\end{algorithmic}
\end{algorithm}


\end{Algoritmos}

\newpage


