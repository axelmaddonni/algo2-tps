\begin{Interfaz}
  
  \textbf{usa}: \tadNombre{}.
  
  \textbf{se explica con}: \tadNombre{Diccionario($\kappa, \sigma$)}.

  \textbf{g\'eneros}: \TipoVariable{diccLog($\kappa, \sigma$)}.

  \Titulos{Operaciones}

  \InterfazFuncion{Definido?}{\In{d}{diccLog($\kappa, \sigma$)}, \In{c}{$\kappa$}}{Bool}
  {res $\igobs$ def?(c,d)}%
  [O(log(n))]

  \InterfazFuncion{Significado}{\In{d}{diccLog($\kappa, \sigma$)}, \In{c}{$\kappa$}}{$\sigma$}
  [def?(c,d)]
  {res $\igobs$ obtener(c,d)}%
  [O(log(n))]
  
  \InterfazFuncion{Vacio}{}{diccLog($\kappa, \sigma$)}
  {res $\igobs$ vacio()}%
  [O(1)]
  
  \InterfazFuncion{Definir}{\Inout{d}{diccLog($\kappa, \sigma$)}, \In{c}{$\kappa$}, \In{s}{$\sigma$}}{}
  [$\neg$def?(c,d) $\wedge$ d=d$_0$]
  {d=definir(c,d$_0$)}%
  [O(log(n))]
  
  \InterfazFuncion{Borrar}{\Inout{d}{diccLog($\kappa, \sigma$)}, \In{c}{$\kappa$}}{}
  [def?(c,d) $\wedge$ d=d$_0$]
  {d=borrar(c,d$_0$)}%
  [O(log(n))]

\end{Interfaz}

\newpage 

\begin{Representacion}

\begin{Estructura}{diccLog}[puntero(estr($\kappa, \sigma$)]
   
\begin{Tupla}[estr($\kappa, \sigma$)]
\tupItem{clave}{\TipoVariable{$\kappa$}} \newline \nomoreitems{}
\tupItem{izquierdo}{\TipoVariable{puntero(nodo($\kappa,\sigma$))}} \newline \nomoreitems{}	 
\tupItem{derecho}{\TipoVariable{puntero(nodo($\kappa,\sigma$))}} \newline \nomoreitems{}
\tupItem{significado}{\TipoVariable{$\sigma$}} \newline \nomoreitems{}
\tupItem{nivel}{\TipoVariable{nat}}
\end{Tupla}

   
\end{Estructura}

Rep: estr $\rightarrow$ bool
$\ $\newline
\newline \indent Rep($e$) $\equiv$ true $\iff$ (($\forall n_1, n_2:\ estr(\kappa, \sigma))(\ n_1 \in\ arbol(e) \wedge \ n_2 \in\ arbol(e) \Rightarrow_L\
\newline \indent (n_1.clave < n_2.clave \Rightarrow n_1 \in arbol(n_2.izquierdo)) \wedge\ (n_1.clave > n_2.clave \Rightarrow n_1 \in arbol(n_2.derecho)) \wedge\ 
\newline \indent ((\forall n_1, n_2:\ estr(\kappa, \sigma))(\ n_1 \in\ arbol(e) \wedge \ n_2 \in\ arbol(e) \wedge\ n_1.clave\neq n_2.clave \Rightarrow_L\
\newline \indent (n_1.izquierdo = n_2.izquierdo \vee n_1.izquierdo = n_2.derecho \Rightarrow n_1.izquierdo = NULO) \wedge\ 
\newline \indent (n_1.derecho = n_2.izquierdo \vee n_1.derecho = n_2.derecho \Rightarrow n_1.derecho = NULO)) \wedge\ 
\newline \indent ((\forall n:\ estr(\kappa, \sigma))(\ n \in\ arbol(e) \wedge n.izquierdo = NULO \wedge n.derecho = NULO \Rightarrow\ n.nivel = 1) \wedge
\newline \indent ((\forall n:\ estr(\kappa, \sigma))(\ n \in\ arbol(e) \wedge n.izquierdo \neq NULO \Rightarrow\ (n.izquierdo).nivel = n.nivel-1) \wedge
\newline \indent ((\forall n:\ estr(\kappa, \sigma))(\ n \in\ arbol(e) \wedge n.derecho \neq NULO \Rightarrow
\newline \indent ((n.derecho).nivel = n.nivel-1 \vee\ (n.derecho).nivel = n.nivel)) \wedge
\newline \indent ((\forall n:\ estr(\kappa, \sigma))(\ n \in\ arbol(e) \wedge n.derecho \neq NULO \wedge_L (n.derecho).derecho \neq NULO \Rightarrow_L
\newline \indent ((n.derecho).derecho).nivel < n.nivel) \wedge
\newline \indent ((\forall n:\ estr(\kappa, \sigma))(\ n \in\ arbol(e) \wedge n.nivel > 1 \Rightarrow\ (n.izquierdo \neq NULO \wedge n.derecho \neq NULO)$
$\ $\newline$\ $
\newline \indent Abs: $dicc(\kappa, \sigma)\ d \rightarrow\ dicc(\kappa, \sigma)$ \{Rep($d$) \}
$\ $\newline
$\ $\newline \indent Abs($d$) $\equiv$ c: dicc($\kappa, \sigma$) | $((\forall k:\ \kappa)(\ k \in claves(c) \Rightarrow Definido?(c, k)) \wedge
\newline \indent (\ Definido?(c, k) \Rightarrow\ k \in claves(c))) \wedge_L
\newline \indent (\forall k:\ \kappa)(\ k \in claves(c) \Rightarrow_L obtener(c,k) =_{obs} Significado(d, k)$

$\ $\newline$\ $
\newline \indent \textbf{Especificaci\'on de las operaciones auxiliares utilizadas para Rep y Abs}
\newline \indent arbol:$\ puntero(estr(\kappa\, \sigma)) \rightarrow conj(\kappa)$

\tadAxioma{arbol(n)}{
\IF {$n.izq \neq null \wedge n.der \neq null$}
THEN {Ag($n.clave,arbol(n.izq)\cup arbol(n.der)$)}
ELSE { \IF {$n.izq \neq null$}
		THEN {Ag($n.clave,arbol(n.izq)$)}
		ELSE { \LIF\ $n.der \neq null$ \LTHEN\ Ag($n.clave,arbol(n.der)$) \LELSE\ Ag($n.clave,\emptyset$) \LFI}
		FI}
FI}

\end{Representacion}

\newpage

\begin{Algoritmos}

\begin{algorithm}
\caption{Implementaci\'on de Definido?}
\begin{algorithmic}[0]
\Function{iDefinido?}{in d: estr , in c: $\kappa$}{$\rightarrow$ res: bool}
	\State nodoActual $\gets$ d
	\State res $\gets$ FALSE
	\While{$\neg$(nodoActual == NULO) $\&\&$ $\neg$res}
		\If{nodoActual.clave == c}
			\State res $\gets$ TRUE
		\Else
			\If{c < nodoActual.clave}
				\State nodoActual $\gets$ nodoActual.izquierdo
			\Else
				\State nodoActual $\gets$ nodoActual.derecho
			\EndIf
		\EndIf
	\EndWhile
\EndFunction
\end{algorithmic}
\end{algorithm}

\begin{algorithm}
\caption{Implementaci\'on de Significado}
\begin{algorithmic}[0]
\Function{iSignificado}{in d: estr , in c: $\kappa$}{$\rightarrow$ res: $\sigma$}
	\State nodoActual $\gets$ d
	\While{$\neg$(nodoActual == NULO) $\&\&$ $\neg$res}
		\If{nodoActual.clave == c}
			\State res $\gets$ nodoActual.significado
		\Else
			\If{c < nodoActual.clave}
				\State nodoActual $\gets$ nodoActual.izquierdo
			\Else
				\State nodoActual $\gets$ nodoActual.derecho
			\EndIf
		\EndIf
	\EndWhile
\EndFunction
\end{algorithmic}
\end{algorithm}

\begin{algorithm}
\caption{Implementaci\'on de Vacio}
\begin{algorithmic}[0]
\Function{iVacio}{}{$\rightarrow$ res: estr}
	\State res $\gets$ NULO
\EndFunction
\end{algorithmic}
\end{algorithm}

\begin{algorithm}
\caption{Implementaci\'on de Definir}
\begin{algorithmic}[0]
\Function{iDefinir}{inout d: estr , in c: $\kappa$, in s: $\sigma$}{}
	\If{d == NULO}
		\State $res\gets \ <c, NULO, NULO, s, 1>$
	\ElsIf{$c < d.clave$}
		\State $d.izquierdo\gets \ iDefinir(d.izquierdo,c,s)$
	\ElsIf{$c > d.clave$}
		\State $d.derecho\gets \ iDefinir(d.derecho,c,s)$
	\EndIf
	\State $d \gets\ Torsion(d)$
	\State $d \gets\ Division(d)$
	\State $res\gets \ d$
\EndFunction
\end{algorithmic}
\end{algorithm}

\begin{algorithm}
\caption{Implementaci\'on de Torsion}
\begin{algorithmic}[0]
\Function{iTorsion}{in d: estr}{$\rightarrow$ res: estr}
	\If{d == NULO $\|$ d.izquierdo == NULO}
		\State res $\gets$ d
	\Else
		\If{(d.izquierdo).nivel $\geq$ d.nivel}
			\State nodoAux $\gets$ d.izquierdo
			\State d.izquierdo $\gets$ nodoAux.derecho
			\State nodo.derecho $\gets$ d
			\State res $\gets$ nodoAux
		\Else
			\State res $\gets$ d
		\EndIf
	\EndIf
\EndFunction
\end{algorithmic}
\end{algorithm}

\begin{algorithm}
\caption{Implementaci\'on de Division}
\begin{algorithmic}[0]
\Function{iDivision}{in d: estr}{$\rightarrow$ res: estr}
	\If{d == NULO $\|$ d.derecho == NULO $\|$ d.derecho.derecho == NULO}
		\State res $\gets$ d
	\Else
		\If{(d.derecho.derecho).nivel == d.nivel}
			\State nodoAux $\gets$ d.derecho
			\State d.derecho $\gets$ nodoAux.izquierdo
			\State nodoAux.izquierdo $\gets$ d
			\State nodoAux.nivel$++$
			\State res $\gets$ nodoAux
		\Else
			\State res $\gets$ d
		\EndIf
	\EndIf
\EndFunction
\end{algorithmic}
\end{algorithm}

\FloatBarrier
\end{Algoritmos}

\begin{algorithm}
\caption{Implementaci\'on de Borrar}
\begin{algorithmic}[0]
\Function{iBorrar}{inout d: estr, in c: $\kappa$}{}
	\If{$d == NULO$}
		\State $endFunction$
	\ElsIf{$c > d.clave$}
		\State $d.derecho\gets \ iBorrar(d.derecho, c)$
	\ElsIf{$c < d.clave$}
		\State $d.izquierdo\gets \ iBorrar(d.izquierdo, c)$
	\ElsIf{$d.izquierdo == NULO \wedge\ d.derecho == NULO$}
		\State $d\gets\ NULO$
	\ElsIf{$d.izquierdo == NULO$}
		\State $aux\gets\ d.derecho$
		\While{$aux.izquierdo \neq NULO$}
			\State $aux\gets\ aux.izquierdo$
		\EndWhile
		\State $d.derecho\gets\ iBorrar(aux.clave, d.derecho)$
		\State $d.clave\gets\ aux.clave$
	\Else
		\State $aux\gets\ d.izquierdo$
		\While{$aux.derecho \neq NULO$}
			\State $aux\gets\ aux.derecho$
		\EndWhile
		\State $d.izquierdo\gets\ iBorrar(aux.clave, d.izquierdo)$
		\State $d.clave\gets\ aux.clave$
	\EndIf
	\State $d\gets\ Nivelar(T)$
	\State $d\gets\ Torsion(T)$
	\If{$d.derecho \neq NULO$}
		\State $(d.derecho).derecho\gets\ Torsion((d.derecho).derecho)$
	\EndIf
	\State $d\gets\ Division(T)$
	\State $d.derecho\gets\ Division(d.derecho)$
	\State $res\gets\ d$
\EndFunction
\Procedure{Nivelar}{inout d: estr}{}
	\State $nivel\_correcto\gets\ min((d.izquierdo).nivel, (d.derecho).nivel)+1$
	\If{$nivel\_correcto < d.nivel$}
		\State $d.nivel\gets\ nivel\_correcto$	
		\If{$nivel\_correcto < (d.derecho).nivel$}
			\State $(d.derecho)nivel\gets\ nivel\_correcto$
		\EndIf
	\EndIf
\EndProcedure
\end{algorithmic}
\end{algorithm}


