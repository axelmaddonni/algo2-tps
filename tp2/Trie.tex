\begin{Interfaz}
  
  \textbf{usa}: \tadNombre{}.
  
  \textbf{se explica con}: \tadNombre{Diccionario(STRING, $\sigma$)}.
  
  \textbf{g\'eneros}: \TipoVariable{diccUniv($\kappa, \sigma$)}.

  \Titulos{Operaciones}

  \InterfazFuncion{Definida}{\In{d}{diccUniv(STRING, $\sigma$)}, \In{c}{STRING}}{Bool}
  {res $\igobs$ def?(c,d)}
  [O(L), donde L es la cantidad de caracteres de la clave m\'as grande.]
  [Indica si la clave dada est\'a definida en el diccionario.]

  \InterfazFuncion{Obtener}{\In{d}{diccUniv(STRING, $\sigma$)}, \In{c}{STRING}}{$\sigma$}
  [def?(c,d)]
  {res $\igobs$ obtener(c,d)}
  [O(L)]
  [Devuelve el significado asociado a la clave dada.]
  [Devuelve al significado por alias.]
  
  \InterfazFuncion{vacio}{}{diccUniv(STRING,$\sigma$)}
  {res $\igobs$ vacio()}
  [O(1)]
  [Crea un diccionario vac\'io.]
  
  \InterfazFuncion{definir}{\Inout{d}{diccUniv(STRING,$\sigma$)}, \In{c}{}STRING, \In{s}{$\sigma$}}{Bool}
  [$\neg$def?(c,d) $\wedge$ d=d$_0$]
  {$d = definir(c,d_0)$}
  [$O(L)+O(copiar(\sigma)$]
  [Agrega la clave al diccionario, asoci\'andole el significado dado como par\'ametro. $res$ indica si la clave ya estaba definida.]
  
  \InterfazFuncion{borrar}{\Inout{d}{diccUniv(STRING,$\sigma$)}, \In{c}{STRING}}{Bool}
  [d=d$_0$]
  {d=borrar(c,d$_0$)}
  [$O(L)+O(borrar(\sigma)$]
  [Borra la clave dada y su significado del diccionario. $res$ indica si la clave estaba definida (su valor es $true$ en caso de estarlo).]
  
  \InterfazFuncion{claves}{\In{d}{diccUniv(STRING,$\sigma$)}}{conj(STRING)}
  {res $\igobs$ claves(d)}
  [O(n*L), donde n es la cantidad de claves.]
  [Devuelve el conjunto de las claves del diccionario.]

\end{Interfaz}
